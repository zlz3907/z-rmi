% Created 2015-10-01 Thu 00:31
\documentclass[11pt]{article}
\usepackage{ctex}
\usepackage{tikz}
\usepackage[utf8]{inputenc}
\usepackage[T1]{fontenc}
\usepackage{fixltx2e}
\usepackage{graphicx}
\usepackage{longtable}
\usepackage{float}
\usepackage{wrapfig}
\usepackage{rotating}
\usepackage[normalem]{ulem}
\usepackage{amsmath}
\usepackage{textcomp}
\usepackage{marvosym}
\usepackage{wasysym}
\usepackage{amssymb}
\usepackage{minted}
\usepackage[unicode]{hyperref}
\usepackage{url}
\usepackage{enumitem,letltxmacro}
\usepackage{geometry}
\LetLtxMacro\itemold\item
\renewcommand{\item}{\itemindent.4cm\itemold}
\usetikzlibrary{mindmap,trees}
\tolerance=1000
\geometry{left=2.5cm,right=2.5cm,top=2.5cm,bottom=2.5cm}
\definecolor{mintedbg}{rgb}{0.95, 0.95, 0.95}
\author{Bliss Chung}
\date{\today}
\title{RMI简单集成}
\hypersetup{
 pdfauthor={Bliss Chung},
 pdftitle={RMI简单集成},
 pdfkeywords={},
 pdfsubject={},
 pdfcreator={Emacs 24.4.1 (Org mode 8.3.2)}, 
 pdfborder={0 0 0}, 
 pdflang={English}}
\begin{document}

\pagenumbering{gobble}
\clearpage
\thispagestyle{empty}
\maketitle
\clearpage
\tableofcontents
\newpage
\pagenumbering{arabic}

\section{关于Z-RMI}
\label{sec:orgheadline2}

\subsection{{\bfseries\sffamily TODO} 制作视频演示(配奇迹背景音乐)}
\label{sec:orgheadline1}

\section{下载和安装(Download and Installation)}
\label{sec:orgheadline5}

\subsection{安装方式一}
\label{sec:orgheadline3}
直接下载 \href{https://github.com/zlz3907/z-rmi/raw/master/dist/z-rmi/z-rmi-all-1.0-jars.jar}{z-rmi-all-1.0.jar} 并添加到项目的 \texttt{CLASSPATH} 中。

\href{https://github.com/zlz3907/z-rmi/raw/master/dist/z-rmi/z-rmi-all-1.0-jars.jar}{\url{https://github.com/zlz3907/z-rmi/raw/master/} $\backslash$
      dist/z-rmi/z-rmi-all-1.0-jars.jar}

\subsection{安装方式二(ivy)}
\label{sec:orgheadline4}

\textbf{注:该项目包暂时发布在内网服务器的资源库中,因此本方法暂只能在公司
 内部使用。}

\begin{minted}[frame=leftline,linenos,xleftmargin=30pt,xrightmargin=30pt,numbersep=3pt]{xml}
<!-- 在你的ivy.xml文件里添加下面的内容  -->
<dependency org="com.ztools" name="z-rmi" rev="1.0" />
\end{minted}

\section{相关配置(Configuration)}
\label{sec:orgheadline6}
项目所有的配置文件都在 \texttt{src/conf/} 文件夹下, 系统启动时默认读取
\texttt{src/conf/configuration.cfg} 文件,这个文件是配置文件的入口,也可以
在启动时以参数的形式指定该文件的物理路径。其它配置文件的路径都可以在
这个文件里进行配置和修改。

\begin{minted}[frame=leftline,xleftmargin=28pt,xrightmargin=28pt]{text}
src/conf
├── configuration.cfg  # 配置文件入口
├── rmi-handler.xml    # RMI主机地址、端口和服务名
└── z-rmi.cfg          # RMI服务端任务执行引擎相关配置,可以自行扩展
\end{minted}

配置文件都可以用默认的,扩展自己的远程方法时只要按照下面的示例在
\texttt{z-rmi.cfg} 添加相应的内容就可以了:

\begin{minted}[frame=leftline,linenos,xleftmargin=30pt,xrightmargin=30pt,numbersep=3pt]{sh}
# 三行配置
com.ztools.rmi.client.IHandlerFactory = com.ztools.rmi.client.CfgHandlerFactory
rmi.handler = conf/rmi-handler.xml
remote.executor.example = com.ztools.rmi.executors.Example

## registry executors
## 注意事项
#  1. 在这里可以配置你想公开给远程客户端的调用的类;
#  2. 引擎只会把公有方法注册到服务列表中供客户端调用;
#  3. 目前暂不支持一个类中有相同的方法名;
#  4. 注册的类应该有一个默认的构造函数;
#
## 配置格式
#  <前缀>.<唯一标识> = <完整的类名>
#
#  <前缀>只能是“remote.executor”
#  <唯一标识>可以是任意值,只要不重复就行,但要复合properties文件的基本要求
#  <完整的类名>是你想要公开给远程客户端调用的类名
#
## 示例
#  remote.executor.string = java.util.String
\end{minted}

\section{使用(Usage)}
\label{sec:orgheadline9}

\subsection{启动服务}
\label{sec:orgheadline7}
命令行启动:

\begin{minted}[frame=leftline,linenos,xleftmargin=30pt,xrightmargin=30pt,numbersep=3pt]{sh}
java -cp z-rmi-all-1.0.jar:yourlib/* com.ztools.rmi.service.RmiService
\end{minted}

或:

\begin{minted}[frame=leftline,linenos,xleftmargin=30pt,xrightmargin=30pt,numbersep=3pt]{sh}
java -jar z-rmi-all-1.0.jar -s
\end{minted}

\subsection{客户端远程方法调用}
\label{sec:orgheadline8}
java调用示例:

\begin{minted}[frame=leftline,linenos,xleftmargin=30pt,xrightmargin=30pt,numbersep=3pt]{java}
public static void main(String[] args) {
  RmiClient client = new RmiClient();
  try {
    String remoteHandler = "com.ztools.rmi.executors.Example";
    // 调用远程方法并等待服务器返回结果,断线会重新连接尝试连接
    client.execute(new Task(remoteHandler, "sayHello"), true);

    // 调用远程方法,如果服务器异常不会重新连接,直接返回
    client.execute(new Task(remoteHandler, "print", "Hello"));
  } catch (Exception e) {
    // TODO Auto-generated catch block
    e.printStackTrace();
  }
}
\end{minted}
\end{document}
